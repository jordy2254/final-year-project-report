\subsection{How is location calculated}
There exists 2 different methods of finding someone’s position the first is a Classification system and the second is co-ordinate based system. A classification based system is a system that can detect the room someone is in and not there exact location whereas a co-ordinate based system is a system that can accurately detect the location of the user.
The focus of the software being produced is to be able to give someone directions to a dedicated location. For this reason it’s important to consider technologies that enable a height range of accuracy. Although if a technology can be found that can be implemented to classify a user is within a space this could come in useful for smaller rooms where accuracy is not a key point such as an office, or classroom.
Calculating a co-ordinate based location is done the same for every technology listed below. This method is called trilateration. Trilateration not to be confused with triangulation, is method of gathering distances from “points” and then finding the intersecting area of those overlapping distances to find the user’s “real location”. The diagram below demonstrates this.

\subsubsection{Increasing accuracy of distance measurements}
Several of the available technologies listed below rely on the same basic principal of measuring a signal strength. Unfortunately signal strength can suffer issues with accuracy due to a range of issues from localised interference on the sensor. This could background noise, something blocking the sensor and more. In addition different devices on the same standard can broadcast at a different strength which can cause issues with calculating the distance from the sensor as each sensor may need to be calibrated prior to being installed withing the system. With a range of different qualities of signal based on the user’s environment and device, it’s important to be able to calibrate sensors to make them more accurate for the user. There exists many papers on this topic and all of them involve calibrating at known distances and the interpolating signal strength from the sensor to the device. 
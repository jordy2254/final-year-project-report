\subsection{How is location calculated}
The following sections aim is to cover the different way location is calculated independent of the technologies required to gather the data. This will give an insight into the algorithms and general approach to positioning systems such that the system can be implemented. 

\subsubsection{Location classification methods}
There exists two key types of location system, the first is a classification system \citetemp this kind of system can classify a zone/area the user is in but not a co-ordinate of their location within that space. An example of this kind of system could be....
The second kind of system is a co-ordinate based system \citetemp, this method of positioning gives an accurate co-ordinate of the user within space. An example implementation of this is GPS which is used to calculate a longitude and latitude of someone on earth \citetemp.

With the projects aims and objective to be able to achieve an accurate and robust indoor positioning system, it's important to consider the technologies and algorithms which will make this a possibility, hence a focus on coordinate based technologies will be made throughout the rest of the research. From a cost perspective it's also important to consider the use of classification for area's where accuracy may not be key, this could be fitting for small area's. Achieving this would require less sensor's and the code developed could also adapt, due to requiring less sensor's it's possible for this to reduce the cost of implementation and therefore making it more accessible.

\subsubsection{General Approach to Location Calculation}
All of the technologies covered within the technology comparison use the same general approach for calculation of the user's location and this is by using lateration. Lateration is the method of measuring distances from sensor's at known location \citetemp, adding more sensor's to this enables a more accurate reading. The calculation of the position from the recieved data is a problem of finding the intersecting area/point of N circles. The most popular method found within research is trilateration which is 3 distance measurements. \citetemp explains that 3 circles is the minimum to achieve point accuracy as anything below this would result in 2 or more possible locations of the user. Below is a diagram demonstrating this principal visually.

\begin{center}
	\includegraphics[width=10cm]{example-image-a}\\
	Image demonstrating circle intersection for 2d location finding from \citetemp
\end{center}

\subsubsection{Increasing accuracy of distance measurements}
All of the technologies covered within the technology comparison rely on the same basic principal of measuring signal strength in order to calculate the distance from the sensor. \citetemp research explains that there can be a range of issues with sensor's and noise, this can stem from a range of environmental issues such as background noise, a blocked sensor etc \citetemp. In addition having a wide range of devices broadcasting in the same spectrum can result in further background noise and interference \citetemp. Finally the broadcast and signal strength provided by devices can change even though they run the same standard, as such calibration for sensor's is important to be able to gain an accurate measurement. \citetemp research explains that a method for doing this is by using linear interpolation with a known/several known calibration points. Furthering from this research paper mobile devices all have varying hardware quality and chips, as such it's important to consider this during application development and give the user the option to calibrate their device. Although many resources such as the prior mentioned research and \citetemp cover the use of linear interpolation for increasing the accuracy in this way it's not clear from the research if calibration cover's the same brand of sensor/chip or if this process needs to be done individually for each sensor and device.\\ 

Angle Of Arrival/Angle Of Departure (AoA/AoD) allows for the use of angulation which is another way of calculating a user's location which instead relies on the measurements/calculation of the received angle. Using AoA/AoD for positioning provides a great advantage compared to either of the technologies alone, yielding more accurate results. \citetemp cover's the use of AoA within bluetooth 5 and how much more accurate the result is from the combination of the technologies. This applies for other technologies discussed below where ones with AoA support offer a superior accuracy compared to those that don't.

\subsubsection{Other ways of calculating location}
On top of the above methods of calculation there are also other methods found that can be used for indoor positioning the first of these is by using a devices accelerometer, an interesting paper from \citetemp cover's this system in detail where by accuracy was concluded to be acceptable. The system works by initialising to a known location and then displacing that location based on the read values from the accelerometer. One large consideration for this technology as found within \citetemp research is the more time that passes from the initialisation of the known co-ordinate the more inaccurate the system is, to get around this the user could be prompted periodically to update their location on the map to maintain accuracy. One method of achieving this within an application is to use QR codes at known position the user scans within apps. Although no paper's exist about combining this technology with other's it could be viable to use this technology in conjunction with the method above. The combination of the methods would allow a feedback system meaning the "initilisation" point of the accelerometer can be regularly update, but also giving a point of truth that can be used to discard bad sensor readings originating from noise or other issues, therefore giving an extra layer accuracy to the system.


Finally there also exists other systems that can be used for indoor positioning however they have large pitfalls not making them viable for this project such as cost, accuracy, dedicated server software etc. some of these technologies include Infrared \citetemp, Ultrasonic \citetemp, RFID \citetemp. 

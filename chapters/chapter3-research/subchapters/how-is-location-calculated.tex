\subsection{How is location calculated}
The following sections aim is to cover the different way location is calculated independent of the technologies required to gather the data. As well as provide an insight into the algorithms and general approach to positioning systems such that the system can be implemented. 

\subsubsection{Location classification methods}
There exists two key types of location system, the first is a classification system \citetemp this kind of system can classify a zone/area of an entity but not a co-ordinate of their location within that space. An example of this kind of system could be....
The second kind of system is a co-ordinate based system \citetemp, this method of positioning gives an accurate co-ordinate of the entity within space, depending on the system this could be a global longitude and latitude such as a GPS system or it could be a localised space such as with an indoor positioning system.

With the projects aims and objective to be able to achieve an accurate and robust indoor positioning system. An important factor to consider is the technologies and algorithms which will make this a possibility, hence a focus on coordinate based technologies will be made throughout the rest of the research. The cost perspective is also significant to consider and using a classification system would allow the cost be reduced, since this kind of system would require one sensor for a given area. Additionally, this system could be implemented into smaller areas where exact location is not required, for example a small office. On the other hand, if there are many small zones that need to be classified near to each other it may become more cost effective keep full positioning as more sensor's may be needed to classify individual offices, but this depends on the implementation details of the classification system.  

\subsubsection{General Approach to Location Calculation}
All of the technologies covered within the technology comparison makes use of lateration for calculation of the users location. Lateration is the method of measuring distances from sensor's at known location \citetemp. The calculation of the position from the recieved data is a problem of finding the intersecting area/point of N circles. The most popular method found within research is trilateration which is 3 distance measurements. \citetemp explains that 3 circles is the minimum to achieve point accuracy as anything below this would result in 2 or more possible locations of the user. Furthermore, using more than 3 sensor's has the ability to achieve a greater accuracy within the calculation, this is because there's more point's of truth so more scope to discard and detect invalid distance measurements. A visual demonstration of the circle intersection for exact point finding can be found below.

\begin{center}
	\includegraphics[width=10cm]{example-image-a}\\
	Image demonstrating circle intersection for 2d location finding from \citetemp
\end{center}

\subsubsection{Increasing accuracy of signal strength based distance measurements}
The most common way distance is calculated for the technologies below is by mesuring the recieved signal strength. \citetemp research explains that there can be a range of issues linked to sensor's and noise, which can range from various environmental issues such as background noise or a blocked sensor. In addition, having a wide range of devices broadcasting within the same spectrum can result in further interference \citetemp. Finally, the quality of the  signal strength can vary independently based on the manufacturer of the device or micro chip, even when running they adhere to the same standard. Additionally, antenna quality/size and broadcast strength are some of the variables that can also effect the perceived signal strength. \citetemp research explains that the accuracy of the distance calculation can be improved by using linear interpolation with a known/several known calibration points. From this research it's been found that it's important to give the user the option to calibrate their device on the network due to the varying hardware found within mobile devices, however it's not clear if this process would be required for identical models of devices or sensors.\\

\subsubsection{Angulation}
Another method of calculating location is by using angulation, within one method of this within wireless signal approaches is the use of Angle Of Arrival/Angle Of Departure (AoA/AoD). These methods are only supported by a limited range of standards such as Ultra-wide band and Bluetooth. These methods can work in conjunction with the lateration approaching yielding a significantly more accurate result, \citetemp cover's this principal with Bluetooth 5 and how to use angulation to get the accuracy down to xcm. .

\subsubsection{Other ways of calculating location}
In addition to the methods above there are other ways of indoor positioning. One of the more interesting methods of this is the use of the phones accelerometer as covered by \citetemp. The paper covers in detail the use of the accelerometer the readings obtained and the accuracy, of which the accuracy was concluded to be acceptable. However, over time the accuracy of the system degrades. The system works by using a known initialisation point of the user and calculating a movement based on the accelerometer readings. Other paper's where also found on this method such as \citetemp which backs up the first paper's findings. To solve the problem of accuracy degradation over time there is a few solutions, one such solution is to include QR codes at regular points around the real indoor location, when the accuracy is known to be too degraded the user can be prompted to scan one of these codes which will update the system. Another approach all though no paper's exist for this could be to use a lateration based approach in conjunction with this method. The combination of the methods would allow a feedback system meaning the initialisation point of the accelerometer system can be regularly updated with no user interaction therefore keeping accuracy. As a result this would provide a great benefit to the main system by also giving a point of truth that can be used to discard bad sensor readings originating from noise or other issues, therefore giving an extra layer accuracy to the system.


Finally, there also exists other systems that can be used for indoor positioning however they have large pitfalls not making them viable for this project aims and objectives. Some of these include cost, accuracy and requiring a dedicated server. These technologies include Infrared \citetemp, Ultrasonic \citetemp, RFID \citetemp. 

\section{Design}
In order to make the application easy to use, design considerations have come from various existing map applications such as google maps and apple maps. These systems have heavily impacted the UI design of the application, so it can be familiar to user's coming from those platforms, making the system easy to use.

The design of the system happened in several in stages which are detailed below.

\subsection{User stories}
User stories are an excellent way of creating more detailed requirements for development. The below user stories are based of the requirements list, each user story contains it's own set of acceptance testing as such will be used in conjunction with the test plan to create a solid foundation for testing the project.

\begin{itemize}
	\item As a business I want to be able to create indoor maps so that my clients can navigate on my premises
	\item As a business I want to be able to password protect my created maps, so only those who have the password (authorised) can synchronise them
	\item As a business I want to be able to view statistics of my maps such as syncronised count, so that I can gauge who's using the system\\
	\item As a consumer I want to synchronise maps so I can get directions in unknown locations
	\item As a consumer I want the path finding to be accurate so that I don't continue to get lost
	\item As a consumer I want to be able to amend the path finding algorithm so that it takes me up lifts as opposed to stairs
	\item As a consumer I want to be able to explore a location prior to arrival to gain an idea of the location
\end{itemize}

\subsection{Test plan}
Extra testing may be needed in conjunction to the scenarios mentioned in the above user stories, as such a full test plan has been developed in order to test the application fully. This test plan will be completed at the end of development, and test cases picked off by feature, to ensure the development done passes the situations layed out.

\subsection{Architecture overview}
With the system being modular and containing several key parts that need to function appropriately together and architecture overview diagram has been generated to show how the system intends to work. This architecture overview is useful for understanding the base system that has been implemented. It can be seen there are 3 main sections

\begin{itemize}
	\item Web application
	\item Android Application
	\item Rest API
\end{itemize}

Each of these sections alone focus as one large "module" 

\subsection{Data Storage}
Before development and to save times it's important to consider any data that may be required on the REST API. Using an Object Relation Manager (ORM) for handling of data is important for large projects and scalability. As an ORM is independent of SQL so the SQL dialect can be changed in the future if ever needed. An ORM's job is to also create the basic table structure and other elements for you. Although an ORM is being used it's still important to know the data that's required and the final structure of the data. Below is several database design standard showing how the final data structure should look.
\\
Some diagrams go here
\\
As can be seen from the data structure it's reasonably straight forward but contains all of the data that's required to store basic map information. The generated data structure within GoLang can be viewed in \appendixtemp

I don't know an erd I guess and data dictionary, maybe some other database design things here too. discuss the choices as well.

\subsection{Sequence Diagrams}
\subsection{Web Application}

\subsection{Android application}

\subsection{UI Designs}
There exists a range of map applications already implemented including Google maps, Waze \& apple Maps. Each mapping application has a similar UI but they do vary slightly. \appendixtemp contains some screenshots of each of these applications. With large companies having the budget to design optimum and efficient UI's the final product of the mapping system will be a combination of these UI's. Below is the intended UI designs for the applications each of these designed where made on Figma, a tool which can be used for pro-typing UI and also linking frames together to end up with a rough interactive version of the design.

\subsubsection{Android Application}


\subsubsection{Web Application}

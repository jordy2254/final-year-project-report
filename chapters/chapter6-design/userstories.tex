\subsection{User stories}
User stories are an excellent way of creating more detailed requirements for development. The below user stories are based of the requirements list, each user story contains it's own set of acceptance testing as such will create a solid test plan to test the final product on for each requirement\\

%Story 1
\begin{longtable}{| p{0.5\textwidth} | p{0.5\textwidth}|}
	\hline
	\newline Story Name & \newline Web app Map editing\\\hline
	\newline Requirements covered: & \newline \ref{web:sso} \& \ref{web:nmap} \& \ref{web:mapedit} \ref{web:roomdec} \\\hline
	\multicolumn{2}{| p{1\textwidth} |}{
		\newline As a Business
		I want to be able to create/edit indoor maps
		So that my clients can navigate on my premises}\\\hline
	\multicolumn{2}{| p{1\textwidth} |}{\begin{center}
			Acceptance Criteria
		\end{center}}\\\hline
	%A/C 1
	\newline\underline{Scenario 1}\newline
	Given user logged in \newline
	When on map selection screen \newline 
	Then only that user's maps are available\newline
	
	&
	
	%A/C 2
	\newline\underline{Scenario 2}\newline
	Given user logged in \newline
	When new map option is selected \newline 
	Then a blank map will be created.\newline
	Then all options required to create a map will be available.\newline
	\\\hline
	%A/C 3
	\newline\underline{Scenario 3}\newline
	Given user logged in \newline
	When existing map is selected \newline 
	Then map editing system will start with the selected map\newline
	Then all options required to create a map will be available.\newline
	&
	\\\hline
\end{longtable}

%Story 2
\begin{longtable}{| p{0.5\textwidth} | p{0.5\textwidth}|}
	\hline
	\newline Story Name & \newline Web: Data loss prevention\\\hline
	\newline Requirements covered: & \newline \ref{web:datalossprevention}\\\hline
	\multicolumn{2}{| p{1\textwidth} |}{
		\newline As a Business 
		I want to be able to be aware of any errors when creating maps
		So that I can be aware of any potential data loss}\\\hline
	\multicolumn{2}{| p{1\textwidth} |}{\begin{center}
			Acceptance Criteria
	\end{center}}\\\hline
	
	\newline\underline{Scenario 1}\newline
	Given user is editing map\newline
	When API is in accessable or returns an error\newline 
	Then user at minimum user is notified of lost data\newline
	Then user has an option to re-try request\newline
	Then user has an option to revert request\newline
	&
	\\\hline
\end{longtable}

%Story 3
\begin{longtable}{| p{0.5\textwidth} | p{0.5\textwidth}|}
	\hline
	\newline Story Name & \newline Android \& web: Performance\\\hline
	\newline Requirements covered: & \newline \ref{web:performance} \ref{android:rendering}\\\hline
	\multicolumn{2}{| p{1\textwidth} |}{
		\newline As a User of either system 
		I want a smooth expience
		So that I can continue to use the software}\\\hline
	\multicolumn{2}{| p{1\textwidth} |}{\begin{center}
			Acceptance Criteria
	\end{center}}\\\hline
	
	\newline\underline{Scenario 1}\newline
	Given user is editing map\newline
	When any change is made\newline 
	Then map re-renderes frame in a speed as defined in \ref{web:performance} \newline
	&
	\newline\underline{Scenario 1}\newline
	Given user is on android app within a map \newline
	Then screen re-renderes frame in a speed as defined in \ref{android:rendering} \newline
	
	\\\hline
\end{longtable}

%Story 4
\begin{longtable}{| p{0.5\textwidth} | p{0.5\textwidth}|}
	\hline
	\newline Story Name & \newline Web app Map security\\\hline
	\newline Requirements covered: & \newline \ref{web:mapsecurity}\\\hline
	\multicolumn{2}{| p{1\textwidth} |}{
	\newline As a Business 
	I want to be able to password protect my created maps\newline
	So that only those who have the password (authorised) can synchronise them}\\\hline
	\multicolumn{2}{| p{1\textwidth} |}{\begin{center}
			Acceptance Criteria
	\end{center}}\\\hline
	%A/C 1
	\newline\underline{Scenario 1}\newline
	Given user has selected map \newline
	When on map settings are viewed \newline 
	Then user has an option to enter a password\newline
	
	&
	\newline\underline{Scenario 2}\newline
	Given A map is protected by a password \newline
	When When an incorrect password is used on synchronisation \newline 
	Then the map fails to synchronise\newline
	\\\hline
\end{longtable}

%Story 5
\begin{longtable}{| p{0.5\textwidth} | p{0.5\textwidth}|}
	\hline
	\newline Story Name & \newline Web app Dashboard page\\\hline
	\newline Requirements covered: & \newline \ref{web:dashboard} \& \ref{android:locdata}\\\hline
	\multicolumn{2}{| p{1\textwidth} |}{
		\newline As a Business 
		I want to be able to view statistics of my maps such as synchronised count
		so that I can gauge who's using the system}\\\hline
	\multicolumn{2}{| p{1\textwidth} |}{\begin{center}
			Acceptance Criteria
	\end{center}}\\\hline
	%A/C 1
	\newline\underline{Scenario 1}\newline
	Given user is logged in\newline
	When user navigates to dashboard->statistics \newline 
	Then user can see statisitics, and also filter by map\newline
	
	&
	\\\hline
\end{longtable}

%Story 6
\begin{longtable}{| p{0.5\textwidth} | p{0.5\textwidth}|}
	\hline
	\newline Story Name & \newline Android: Syncronisation\\\hline
	\newline Requirements covered: & \newline \ref{android:sync} \& \ref{web:mapsecurity} \& \ref{android:offline} \\\hline
	\multicolumn{2}{| p{1\textwidth} |}{
		\newline As a Consumer 
		I want to be able to Synchronise maps easily
		So that I can quickly and efficiently get directions in unknown locations}\\\hline
	\multicolumn{2}{| p{1\textwidth} |}{\begin{center}
			Acceptance Criteria
	\end{center}}\\\hline
	%A/C 1
	\newline\underline{Scenario 1}\newline
	Given user is on main menu \newline
	When plus button is pressed \newline 
	When Valid Map ID \& password are entered \newline 
	When Confirm button is pressed \newline 
	Then new map is added to the list of maps\newline
	
	& 
	\newline\underline{Scenario 2}\newline
	Given user is on main menu \newline
	When plus button is pressed \newline 
	When Invalid Map ID is entered \newline 
	When Confirm button is pressed \newline 
	Then Error message is displayed and map is not added to the list\newline\\\hline
	
	\newline\underline{Scenario 3}\newline
	Given user is on main menu \newline
	When plus button is pressed \newline 
	When Valid map ID is entered \newline 
	When Invalid map password is entered \newline 
	When Confirm button is pressed \newline 
	Then Error message is displayed and map is not added to the list\newline
	
	&
	
	\newline\underline{Scenario 4}\newline
	Given user is on main menu \newline
	Given user has maps synchronised \newline
	When Map delete is pressed \newline 
	Then The map file is deleted from the device and removed from the view\newline\\\hline
	
	\newline\underline{Scenario 5}\newline
	Given user is on main menu \newline
	Given user has maps synchronised \newline
	Given user map is outdated \newline
	When Map synchronisation button is pressed \newline 
	Then The map is resynchronised from the server updating in the UI\newline
	
	&
	
	\newline\underline{Scenario 6}\newline
	Given user is on main menu \newline
	Given user has no maps synchronised \newline
	Then A message is displayed until successful synchronization notifying of this\newline\\\hline
	
	\newline\underline{Scenario 7}\newline
	Given user is on main menu \newline
	Given user has synchronised a map\newline
	Then UI adds the map to the list, and updates the UI\newline
	
	&
	
	\newline\underline{Scenario 8}\newline
	Given user opens there QR code app of choice \newline
	Given QR code details are correct\newline
	When user scans a map QR code\newline
	Then QR code will lead to main page of application\newline
	Then Map will be syncronised\newline
	\\\hline
	
	\newline\underline{Scenario 9}\newline
	Given user is on main menu \newline
	Given user has no internet connection\newline
	Given user has already successfully syncronised map \newline
	Then user can use the application as normal\newline
	&
	\\\hline
\end{longtable}

%Story 5
\begin{longtable}{| p{0.5\textwidth} | p{0.5\textwidth}|}
	\hline
	\newline Story Name & \newline Android: Path finding modifications\\\hline
	\newline Requirements covered: & \newline \ref{android:pathfindingmod}\\\hline
	\multicolumn{2}{| p{1\textwidth} |}{
		\newline As a Consumer 
		I want to be able to avoid stairs
		So that I have the option to use lifts, or access an area as a disabled user}\\\hline
	\multicolumn{2}{| p{1\textwidth} |}{\begin{center}
			Acceptance Criteria
	\end{center}}\\\hline

	\newline\underline{Scenario 1}\newline
	Given user has selected map \newline
	Given user has selected a start and end point \newline
	When on map route options are viewed \newline 
	Then user has an option to avoid stairs\newline
	&
	\newline\underline{Scenario 2}\newline
	Given user has selected avoid stairs option \newline
	Given Map has stairs \& lifts \newline
	When path crosses a staircase \newline 
	Then Path finding will find an alternate route\newline
	\\\hline
\end{longtable}

%Story 6
\begin{longtable}{| p{0.5\textwidth} | p{0.5\textwidth}|}
	\hline
	\newline Story Name & \newline Android: Route Selection\\\hline
	\newline Requirements covered: & \newline\ref{android:locsel} \& \ref{android:pathfinding} \\\hline
	\multicolumn{2}{| p{1\textwidth} |}{
		\newline As a Consumer 
		I want to be able to explore a location prior to arrival
		So that I can gain knowledge of the location prior to arrival}\\\hline
	\multicolumn{2}{| p{1\textwidth} |}{\begin{center}
			Acceptance Criteria
	\end{center}}\\\hline
	%A/C 1
	\newline\underline{Scenario 1}\newline
	Given user has selected map \newline 
	Then user has can drag/zoom around the map \& change which floor is being viewed \newline
	&
	\newline\underline{Scenario 2}\newline
	Given user has selected map \newline 
	Then user can select a target location \newline
	\\\hline
	
	\newline\underline{Scenario 3}\newline
	Given user has selected a target location \newline 
	Then user has can select a start location \newline
	&
	\newline\underline{Scenario 4}\newline
	Given user has selected start \& end locations \newline 
	When user clicks the directions button \newline
	Then user can see path drawn on map \newline
	\\\hline
	\newline\underline{Scenario 5}\newline
	Given user has selected current location for start \newline 
	Given user selected end location, and is in direction mode \newline 
	Then The path will update as the user walks it \newline
	&
	\\\hline
\end{longtable}

%Story 6
\begin{longtable}{| p{0.5\textwidth} | p{0.5\textwidth}|}
	\hline
	\newline Story Name & \newline Android: Route display\\\hline
	\newline Requirements covered: & \ref{android:pathfinding} \& \ref{android:writtendir} \& \ref{android:voicedir} \& \ref{android:posacc}\\\hline
	\multicolumn{2}{| p{1\textwidth} |}{
		\newline As a Consumer 
		I want information on directions
		So that I can gain an understanding of the area}\\\hline
	\multicolumn{2}{| p{1\textwidth} |}{\begin{center}
			Acceptance Criteria
	\end{center}}\\\hline
	%A/C 1
	\newline\underline{Scenario 1}\newline
	Given user has started directions \newline 
	Then user can turn on voice prompts\newline
	&
	\newline\underline{Scenario 2}\newline
	Given user has started directions \newline 
	Then user has can opt to view written directions \newline
	\\\hline
	\newline\underline{Scenario 3}\newline
	Given user is within sensor location \newline 
	When user opens that map \newline
	Then user can accurately see location within distance defined in \ref{android:posacc} \newline
	&
	\\\hline
\end{longtable}




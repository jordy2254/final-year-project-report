\subsubsection{Bluetooth Beacons}
Bluetooth beacons contain 2 main standards these are Eddy-stone \& IBeacon, these standards are developed by google and apple respectively \citetemp. Although they are classified as 2 different standards, they both run on BLE devices and according to several resources such as \citetemp they don’t feature many distinct differences so, for this reason the rest of this section will refer to just beacons as group and not a specific standard of beacons.

With beacons acting as independent unit's with one way communication broadcasting their identification \citetemp this provides a huge advantage that many of the other technologies don't have. This trait is the fact that networking topologies and layout is not a consideration required to install the sensor's. This in turn means that the only real install consideration is the spacing and location of the sensor's in order to achieve optimal results. In turn this also means the beacon system can easily be adapted in order to meet new requirements or reach new areas within the business. This made even easier by there long battery life \citetemp sells beacons that will last up to 58 months with no maintenance or battery changes. This in turn lowers the cost of install and maintenance further and in turn also makes the technology viable for remote area's or locations where cellular data, Wi-Fi or electricity are not available. Finally \citetemp has beacons available from as low as £10 meaning the initial investment of beacons is also lower than that of other technologies with higher costs for the technology. Finally compiling various resource papers such as \citetemp it can be seen the technology has the ability to provide a high level of accuracy making it a viable option for the aims and objectives.

It's been identified from the research the one way nature of Bluetooth beacons and the advantages they offer. From this information it can be established that there is a potential to offer a high level of privacy. Depending on the implementation and how it is designed it could be ran locally on the users device. This provides two advantages, and the first is the cost saving of not requiring a central server to store user's information of calculate their position. The second is that a user may feel more comfortable using the system if their data is not being shared or has the opportunity to not be shared to a central location. \citetemp discusses how people are starting to care more about their online privacy and this implementation could lead to more trust within the application.


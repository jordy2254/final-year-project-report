\subsubsection{Ultra wide band (UWB)}
UWB is a technology that has existed for a long time and is commonly used for precision positioning scenarios. Although it’s a technology that has been around for a long time it’s something that has only just started becoming available within smart phone tech, from the iPhone 11 and Galaxy note 20 ultra in august 2020 \citetemp. In addition to this google has now also started developing a general API as part of the Android SDK \citetemp, this will make development easier and more obtainable for developers. Google being on board and expanding their SDK for future with UWB technology also demonstrates the potential for it becoming a normal part of smartphones like infrared, bluetooth \& Wireless charging etc are today.\\

UWB is responsible for a lot of industrial applications in location tracking such as a company called Sewio. Sewio is a provider of IPS technologies and has some large clients including Volkswagen, Budweiser amongst other’s, which in turn shows support from large companies for the technology. With UWB having implementations in many industry-based applications demonstrates that the technology could be the future of indoor positioning systems. 

UWB like Bluetooth 5.1 has the ability for AOA \citetemp thus combines the advantages of Bluetooth 5.1 however in contrast to Bluetooth 5.1 UWB uses a different radio spectrum this leads to less interference and a stronger signal \citetemp. Unfortunately, unlike Bluetooth based implementations UWB is still an expensive technology to implement, Sewio has prices listed on there website and packages starting at €2850 \citetemp for just 5 sensors/tags. Part of this cost probably comes from the currently small adoption of UWB, leading to a higher implementation cost. As UWB becomes more wide spread like most technologies this cost should theoretically come down.\\

Finally A majority of smartphones not having UWB capability’s is a large draw back for the technology and makes the end goal of this project impossible as it will require dedicated hardware for a majority of user’s. Although  this is currently the case it’s important to design the system in mind of expanding technologies, as such the design of the system should allow easy switching of technologies in the future to reflect improvement of smartphone technologies.

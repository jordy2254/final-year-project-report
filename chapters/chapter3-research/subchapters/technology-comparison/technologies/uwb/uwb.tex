\subsubsection{Ultra wide band (UWB)}
UWB is a technology that has existed for a long time and is commonly cited to be used within indoor positioning systems. It has recently started growing more traction within the tech world with it's implementation within modern smart phones. This recent implementation happened with the iPhone 11 and Galaxy note 20 ultra in august 2020 (\cite{vyas_2021_google}). With it's growing traction within the smartphone as well as the internet of things (IOT) such as the Galaxy Note smart-tag (\cite{samsung_2021_introducing}) it's showing promising results in becoming a normal addition to mobile devices as infrared, Bluetooth and wireless charging are today. In addition to this Samsung have also released an "insights" article about the future of ultra-wide band in their devices (\cite{samsung_2021_what}), this article states that UWB will be used for "digital" keys, which have the ability to unlock compatible cars. Furthering this the system will make use of UWB accurate distance measurements to ensure the car is unlocked when it should be. Finally google have recently been found adding general support into their SDK for UWB technology (\cite{vyas_2021_google}) this further show's support and adoption by larger companies. 

UWB works differently to Bluetooth technologies and doesn't rely on signal strength to calculate the distance. This makes UWB more accurate, an article by \cite{connell_2015_stackpath} show's that UWB instead works by sending out short 2ns pulses and then uses time of arrival to calculate the distance. As identified by the research by \cite{musa_2019_a} (Non-line-of-site (NLOS)) caused by walls and other obstacles leads to an over estimation in the distance due to the extra interference. This paper focusses on a system specifically designed for NLOS of UWB transmission and still keeping the accuracy by using decision trees. \cite{comer_uwb_vs_ble} compares BLE with UWB and the significant advantages that UWB provides in comparison when being used for distance calculation 

There currently exists a few companies which provide UWB technology for indoor positioning within industrial application. One company found is a company called Sewio. This company has a range of clients including Volkswagen, Budweiser amongst other’s. UWB is still currently expensive \cite{sewio_2021_realtime} start their kit's at €2850 for 5 tracking sensor's this is a high price pay for an every day user or small business therefore making unavailable to them. However with the constant adoption and addition into tech this price should reduce. It's well known that as technologies become more popular the price per unit decreases, \cite{tuovila_2021_marginal} article discusses this point highlighting the reasons this happens, and it's down to the cost of production and the added efficiencies of mass production. The same should happen for UWB technologies as their use becomes more popular, further allowing them to be seen within many devices. 

A majority of smartphones not having the capability for UWB within the device provides a drawback for the technology, it also means the technology cannot be implemented to meet the projects aims and objectives of a generic and adaptable system. This being said it appears with the strong adoption in the recent months from a range of large companies UWB will become a common thing to see in smartphones and other devices. For this reason it's important to implement the system with a focus on expanding technologies so the system can remain viable in the future.\\


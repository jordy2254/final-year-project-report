\subsubsection{Bluetooth Beacons}
Bluetooth beacons have a range of papers discussing them for the use of indoor positioning, one such paper is by \cite{kingatua_2020_bluetooth} which concludes that they are a viable option for a positioning system. Bluetooth is a technology that uses RSSI and as  with identified research earlier from \cite{kontakt_ibeacon} will use the Log-normal Distance Path Loss (LDPL) model for calculating the distance from the sensor based on it's signal strength. The remaining information within this section is going to highlight some of the reasons why they are a viable option and some of their pitfalls. 

Bluetooth beacons contain two main standards which are Eddy-stone \& IBeacon, these standards are developed by google and apple respectively \cite{kontakt_ibeacon}. Although they are classified as two different standards, they both run on BLE devices. With research from \cite{kontakt_ibeacon} as well as other sources it's been found the there are not many differences between the two standards as such the rest of this section is going to cover beacons as a group and not a specific standard.

The documentation for beacons such as google's eddystone standards \cite{google_googleeddystone} show's that beacons are design for one way communication as indepenedent units, broadcasting the data they have been configured with. Beacons acting as independent unit's provides a huge advantage that many of the other technologies don't have. This trait directly results in not requiring the considerations for topologies and networking in order for installation like WiFi. This in turn means that the only real install consideration is the spacing and location of the sensor's in order to achieve optimal results. In turn this also means the beacon system can easily be adapted in order to meet new requirements or reach new areas within the business. With beacons having a battery life of up to 58 months \cite{beaconzone_2016_google} it further allows a reduction in long term costs by making maintenance a minimum. This long term battery life also makes them viable for more remote areas where cellular data, WiFi or electricity are not available. Finally \cite{beaconzone_2016_google} has beacons available from as low as £15 meaning the initial investment of beacons is also lower than that of other technologies with higher costs for the technology. Finally compiling various resource papers such as (\cite{zhuang_2016_smartphonebased}), (\cite{kingatua_2020_bluetooth}) it can be seen the technology has the ability to provide a high level of accuracy making it a viable option for the aims and objectives.

It's been identified from the research the one way nature of Bluetooth beacons and the advantages they offer. From this information it can be established that there is a potential to offer a high level of privacy. Depending on the implementation and how it is designed it could be ran locally on the users device. This provides two advantages, and the first is the cost saving of not requiring a central server to store user's information of calculate their position. The second is that a user may feel more comfortable using the system if their data is not being shared or has the opportunity to not be shared to a central location. Research performed by \cite{axel_2018_how} show's that user's are considering the use of the data more than they used to and also understand the value of their data. One statistic from this research is that "75\% aren't concerned about passing over their data however they expect some form of reward in return", the possibility for an anonymous server-less implementation will appeal to the 25\% that are concerned by their data but also to the 75\% making them more comfortable and trusting with the application.


\subsection{How is location calculated}
The following sections aim is to cover the different way location is calculated independent of the technologies required to gather the data. As well as provide an insight into the algorithms and general approach to positioning systems such that the system can be implemented. 

\subsubsection{Location classification methods}
There exists two key types of location system, the first is a classification system \citetemp this kind of system can classify a zone/area of an entity but not a co-ordinate of their location within that space. An example of this kind of system could be....
The second kind of system is a co-ordinate based system \citetemp, this method of positioning gives an accurate co-ordinate of the entity within space, depending on the system this could be a global longitude and latitude such as a GPS system or it could be a localised space such as with an indoor positioning system.

With the projects aims and objective to be able to achieve an accurate and robust indoor positioning system. An important factor to consider is the technologies and algorithms which will make this a possibility, hence a focus on coordinate based technologies will be made throughout the rest of the research. The cost perspective is also significant to consider and using a classification system would allow the cost be reduced, since this kind of system would require one sensor for a given area. Additionally, this system could be implemented into smaller areas where exact location is not required for example a small office. On the other hand it's also crucial to consider the limitation of the area that can be classified. In example if there are multiple offices near to each other it may be more cost effective to keep full positioning as more sensor's may be needed to classify individual offices, depending on the method used.  

\subsubsection{General Approach to Location Calculation}
All of the technologies covered within the technology comparison makes use of lateration for calculation of the users location. Lateration is the method of measuring distances from sensor's at known location \citetemp. The calculation of the position from the recieved data is a problem of finding the intersecting area/point of N circles. The most popular method found within research is trilateration which is 3 distance measurements. \citetemp explains that 3 circles is the minimum to achieve point accuracy as anything below this would result in 2 or more possible locations of the user. Moreover, using more than 3 sensor's has the ability to achieve a greater accuracy within the calculation. Below is a diagram demonstrating this principal visually.

\begin{center}
	\includegraphics[width=10cm]{example-image-a}\\
	Image demonstrating circle intersection for 2d location finding from \citetemp
\end{center}

\subsubsection{Increasing accuracy of distance measurements}
Many of the technologies that make use of lateration for calculating location, rely on the principal of measuring signal strength. \citetemp research explains that there can be a range of issues linked to sensor's and noise, which can range from various environmental issues such as background noise or a blocked sensor. In addition, having a wide range of devices broadcasting within the same spectrum can result in further interference \citetemp. Finally, the quality of the  signal strength can vary independently based on the manufacturer of the device or micro chip, even when running they adhere to the same standard. Additionally, antenna quality/size and broadcast strength are some of the variables that can also effect the perceived signal strength. \citetemp research explains that the accuracy of the distance calculation can be improved by using linear interpolation with a known/several known calibration points. From this research it's been found that it's important to give the user the option to calibrate their device on the network due to the varying hardware found within mobile devices, however it's not clear if this process would be required for identical models of devices or sensors.\\

\subsubsection{Angulation}
Angle Of Arrival/Angle Of Departure (AoA/AoD) is another method that can aid in increased accuracy of location, this method instead relies on the measurements/calculation of the received or departed angle. Using AoA/AoD for positioning provides a great advantage compared to either of the technologies alone, yielding more accurate results. \citetemp cover's the use of AoA within bluetooth 5 and how much more accurate the result is from the combination of the technologies, according to the paper the accuracy of Bluetooth 5 is down to xcm. This applies for other technologies discussed below where ones with AoA support offer a superior accuracy compared to those that don't.

\subsubsection{Other ways of calculating location}
On top of the above methods of calculation there are also other methods found that can be used for indoor positioning the first of these is by using a devices accelerometer, an interesting paper from \citetemp cover's this system in detail where by accuracy was concluded to be acceptable. The system works by initialising to a known location and then displacing that location based on the read values from the accelerometer. One large consideration for this technology as found within \citetemp research is the more time that passes from the initialisation of the known co-ordinate the more inaccurate the system is, to get around this the user could be prompted periodically to update their location on the map to maintain accuracy. One method of achieving this within an application could be by using QR codes at known positions the user then scans this within the app which then re-initialises that initial point. Although no paper's exist about combining this technology with other's it could also be viable to use this technology in conjunction with the positioning methods above. The combination of the methods would allow a feedback system meaning the initialisation point of the accelerometer system can be regularly updated therefore keeping accuracy. Doing this would provide a great benefit to the main system by also giving a point of truth that can be used to discard bad sensor readings originating from noise or other issues, therefore giving an extra layer accuracy to the system.


Finally there also exists other systems that can be used for indoor positioning however they have large pitfalls not making them viable for this project such as cost, accuracy, dedicated server software etc. some of these technologies include Infrared \citetemp, Ultrasonic \citetemp, RFID \citetemp. 

\subsubsection{Bluetooth Beacons}
Bluetooth beacons contain 2 main standards these are Eddy-stone \& IBeacon, these standards are developed by google and apple respectively. Although they are classified as 2 different standards, they both run on BLE devices and according to several resources don’t feature many distinct differences. For this reason the rest of this section will refer to just beacons as group and not a specific standard.

Using beacons contains a few key advantages that other technologies are limited on. The first of these is the lack of networking, Beacons can be individually installed independently of others and work in one direction. The key advantage of this is it enables beacons to work independently of each other, this leads to simplicity in expanding as complicated network topologies are not needed to be considered as each device is standalone and connects to the one device.

This means and increase into user’s privacy as the business does not know the location of the user or where they have been. Having this trait is desirable in this day and age where privacy is a key term and thing looked at by people. Although data can be indirectly read back to a server from an application, this solution can be trusted more?
Another plus side of this technology is a backend server is not needed to maintain and manage everyone’s location. This means a standalone system could be implemented somewhere there isn’t even a wifi connection to begin with. This is ideal for situations where user’s don’t have much mobile data. Or do not have an internet connection.

This not only increases privacy of user’s, but also negates the need for a back-end server to manage the data, or to have an elaborate scalable network in place to support future changes. 
Another upside is beacons is they can be designed with batters and placed in outdoor locations, and periodically replaced. This means that some locations that could be missed will then be covered by the location tracking system. This is particularly important for locations that have more than one building in a small area such as the university campus. This will allow seamless integration and directions from Q Block to the SU and more.
Although beacons have there upside there is some limitations such as lack of accuracy, distance limitations. The distance limitations lead to the need for beacons at a closer density.

Bluetooth beacons also work in a one way relationship, this means that network architecture is not needed and scalability of the system is easier.

\subsubsection{Bluetooth Beacons}
Bluetooth beacons contain 2 main standards these are Eddy-stone \& IBeacon, these standards are developed by google and apple respectively \citetemp. Although they are classified as 2 different standards, they both run on BLE devices and according to several resources such as \citetemp they don’t feature many distinct differences so, for this reason the rest of this section will refer to just beacons as group and not a specific standard of beacons.

Beacons offer one huge advantage that some other technologies don't contain. This huge advantage is the lack of networking required to set them up. As each beacon is an independent unit and each phone reads the signal from the beacon itself, not complicated network topologies are needed to be considered. This also means that an end user can install extra beacons and add them to the mapping system to easily increase accuracy in dead-zones. Many beacons use minimal amounts of power such as the beacons available from \citetemp, which last for a whopping  58 months. Having huge battery lives means that remote locations without/limited accessibility to mains power, means beacons can be installed and periodically up kept.

One more advantage of beacons is because they can run souley of mobile hardward the location algorithm can built into the device to run offline. Not only does this further expand the above point for the possibility to install in remote locations, but it also helps reassure people of privacy as data is kept within the phone and not processed on external servers. Having this trait is desirable in this day and age where privacy is a key term and thing looked at by people. In the event a centralised system is needed the system could easily be adapted to perform these operations and send them back to the server for whatever data storage is needed for them.


This not only increases privacy of user’s, but also negates the need for a back-end server to manage the data, or to have an elaborate scalable network in place to support future changes. 
Another upside is beacons is they can be designed with batters and placed in outdoor locations, and periodically replaced. This means that some locations that could be missed will then be covered by the location tracking system. This is particularly important for locations that have more than one building in a small area such as the university campus. This will allow seamless integration and directions from Q Block to the SU and more.
Although beacons have there upside there is some limitations such as lack of accuracy, distance limitations. The distance limitations lead to the need for beacons at a closer density.

Bluetooth beacons also work in a one way relationship, this means that network architecture is not needed and scalability of the system is easier.

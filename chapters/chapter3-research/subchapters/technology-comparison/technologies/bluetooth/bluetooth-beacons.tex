\subsubsection{Bluetooth Beacons}
Bluetooth beacons have a range of papers discussing them for the use of indoor positioning. One such paper is by \cite{kingatua_2020_bluetooth} which concludes that they are a viable option for a positioning system. Bluetooth is a technology that uses RSSI (Signal strength) and as  with earlier identified research from \cite{kontakt_ibeacon} will use the Log-normal Distance Path Loss (LDPL) model for calculating the distance from the sensor based on it's signal strength. The remaining information within this section is going to highlight some of the reasons why they are a viable option and some of their pitfalls.\\ 

Bluetooth beacons contain two main standards which are Eddy-stone \& IBeacon, these standards are developed by Google and Apple respectively (\cite{kontakt_ibeacon}). Although they are classified as two different standards, they both run on BLE devices. The article by \cite{kontakt_ibeacon} covers the differences between these two standards. Summarising this article the differences are small and don't pose a significant change in the accuracy when being used for indoor positioning. As such, the remaining of this research will refer to beacons as a group as opposed to a specific standard.\\

The documentation for beacons such as Google's Eddy-stone standards (\cite{google_googleeddystone}) shows that beacons are designed for one way communication as independent units, broadcasting the data that they have been configured with. This trait directly results in not requiring the considerations for network topologies and other complicated install considerations that other methods do such as WiFi. The key install consideration identified from research is the spacing and location of the sensors in order to achieve optimal results and accuracy. In turn this also means the beacon system can easily be adapted in order to meet new requirements or reach new areas within the install location. With beacons having a battery life of up to 58 months (\cite{beaconzone_2016_google}) it further allows a reduction in long term costs by making maintenance a minimum. This long term battery life also makes them viable for more remote areas where cellular data, WiFi or electricity are not available. Finally \cite{beaconzone_2016_google} has beacons available from as low as £15 meaning the initial investment of beacons is also lower than that of other technologies with higher costs for the technology. One of the key downsides of Bluetooth beacons is the RSSI based distance measurements, \cite{comer_uwb_vs_ble} covers about signal degradation and how even an obstacle as small as a hand can throw measurements off by metres. Finally, compiling various resource papers such as (\cite{zhuang_2016_smartphonebased}), (\cite{kingatua_2020_bluetooth}) it can be seen the technology has the ability to provide a high level of accuracy making it a viable option for the aims and objectives.\\

Its been identified from the research the one way nature of Bluetooth beacons and the advantages they offer. From this information it can be established that there is a potential to offer a high level of privacy. Depending on the implementation and how it is designed it could be ran locally on the user's device. This provides two advantages, the first is the cost saving of not requiring a central server to store users' information or to calculate their position. The second is that a user may feel more comfortable using the system if their data is not being shared or has the opportunity to not be shared to a central location. Research performed by \cite{axel_2018_how} shows that users are considering the use of the data more than they used to and also understand the value of their data. One statistic from this research is that "75\% aren't concerned about passing over their data however they expect some form of reward in return". The possibility for an anonymous server-less implementation will appeal to the 25\% that are concerned by the data they share, but also to the 75\% making them more comfortable and trusting with the application. Finally, it has also been identified the sensitivity to obstacles, as such, if this technology is chosen the placement of sensors will need to be considered in order to avoid this potential issue.


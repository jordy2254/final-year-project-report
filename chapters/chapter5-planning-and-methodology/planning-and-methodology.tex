\section{Methodology \& Planning}
\subsection{Methodology}
For any kind of development it's important to have a plan of how the software is going to take shape and progress as it goes on. 

This project will be designed in iterative stages with a feedback loop so the project is dynamic and can change direction as research and other area's are finalised. In addition it's important to choose a methodology that is not heavy on design and documentation, having this trait allows the system to be developed quickly. The best suited methodologies for both of these scenarios are Extreme Programming (XP) and Rapid Application Development (RAD). As both of these methodologies are well known and compared online no direct comparison will be made of these, however below are some reference images for the principal of each methodology.

\begin{center}
	\includegraphics[width=7.5cm, height=7.5cm]{example-image-a}\\
	Extreme Programming
\end{center}

\begin{center}
	\includegraphics[width=15cm, height=5cm]{example-image-b}\\
	Rapid Application Development
\end{center}

As mentioned above the chosen methodology needs to have a priority on quick development part of this is due to the limited time of the project overall. RAD is a methodology specifically designed for this use however it does have some well known caveats such as the bad quality of code it produces \citetemp. This is where XP comes in. Being an agile development methodology XP still allows dynamic changing however it also enables a higher quality of code where unit testing and other functionality is also considered as part of the development lifecycle as shown above. Due to this reason and others both methodologies will be used throughout the project. RAD will be used to gain initial prototypes and gain a basic infrastructure. Once this is achieved the methodology will be changed to XP to allow for a better quality of code and more maintainable project for the long run.\\

Both cycles require a feedback loop from stakeholders to allow for changes, both the author and project supervisor will be used for this feedback loop allowing either party to change the overall direction of the final product.

\subsection{Planning}
A project plan has been developed listing the key milestones intended for the project over the year this can be seen below. The below diagram also demonstrates where the transition from RAD to XP happens.

\begin{center}
	\includegraphics[width=7.5cm, height=7.5cm]{example-image-a}\\
	Project plan
\end{center}

\subsubsection{Development Tools}
Version control is important as part of any project, as such a repository will be made in order to manage the project and ensure error's that are introduced can be reverted as well as having a rough log of the work and progress that has happened throughout development. Git is the current industry standard for version control so this system will be used, a remote repository will be stored on a personal github page.\\

There are many IDE's that exist for a range of purposes. The author is most familiar with the Jetbrains Toolsets such as GoLand and Webstorm, Android studio is also based of the community edition of this. As such the Jetbrains projects will be the IDE of choice for development. However it's important to use a build system for each project in order to not tie the project to a specific IDE so the build model can be changed in the future.



\subsubsection{Wifi Signal Strength}
This technology is based on a similar principal to the Bluetooth beacons chapter where the received signal strength is interpolated into a calibrated distance \citetemp. Because of this it shares a lot of the same positive and negative functionality of the Bluetooth technology. Although one key difference is the hardware and implementation differences, Wi-Fi access in larger buildings is normally well established and in some cases the setup may be applicable to implementing an indoor positioning system. However in most cases the overlap of Wi-Fi points will probably not be sufficient meaning further hardware will need to be installed in order to be able to locate a user accurately, this means considerations for network topologies would be required and expert installation/setup would be required increasing the cost of the initial investment. Compared to Bluetooth beacon technologies the cost of a Wi-Fi location tracking is expensive with no real gain in accuracy. Adding onto this the extra overhead of the network, more connected devices to the network results in Wi-Fi location tracking being a bad solution for the particular case.
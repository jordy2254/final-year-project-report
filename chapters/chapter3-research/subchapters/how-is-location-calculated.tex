\subsection{How is location calculated}
The following sections aim is to cover the different way location is calculated independent of the technologies required to calculate the values. This will give an insight into the algorithms and code required to turn the mesurements from sensor's into a final location for the user. 

\subsubsection{Location classification methods}
There exists two key types of location system, the first is a classification system \citetemp this kind of system can classify a zone/area the user is in but not a co-ordinate of their location within that space. An example of this kind of system could be....
The second kind of system is a co-ordinate based system \citetemp, this method of positioning gives an accurate co-ordinate of the user within space. An example implementation of this is GPS which is used to calculate a longitude and latitude of someone on earth \citetemp.

With the aim of this project to produce a positioning system for directions the end result will be a co-ordinate based system. For this reason it's important to consider technologies that enable a high range of accuracy. Although if time permits a duel based system could be implemented where small rooms will feature a classification such as a small office/class room. The implementation of a combined system could reduce the overall cost of Implementing the system therefore making it more accessible.

\subsubsection{General Approach to Location Calculation}
All of the below technologies use the same general approach for calculation of the user's location and this is by using lateration. Lateration is the method of measuring distances from sensor's at known location \citetemp, adding more sensor's to this enables a more accurate reading as the calculation is effectively a problem of finding the intersecting area/point of N circles. As such the most popular method found within research is trilateration which would be 3 measurements. \citetemp explains that 3 circles is the minimum to achieve point accuracy as anything below this would result in 2 or more possible locations of the user. Below is a image demonstrating this principal visually.

\begin{center}
	\includegraphics[width=10cm]{example-image-a}\\
	Image demonstrating circle intersection for 2d location finding from \citetemp
\end{center}

\subsubsection{Increasing accuracy of distance measurements}
Several of the available technologies covered below rely on the same basic principal of measuring signal strength in order to calculate the distance from the sensor. \citetemp research explains that there can be a range of issues with sensor's and noise, this can stem from a range of environmental issues such as background noise, a blocked sensor etc \citetemp. In addition having a wide range of devices broadcasting in the same spectrum can result in further background noise and interference \citetemp. Finally the broadcast and signal strength provided by devices can change even though they run the same standard, as such calibration for sensor's is important to be able to gain an accurate measurement. \citetemp research explains that a method for doing this is by using linear interpolation with a known/several known calibration points. Furthering from this research paper mobile devices all have varying hardware quality and chips, as such it's important to consider this during application development and give the user the option to calibrate their device. Although many resources such as the prior mentioned research and \citetemp cover the use of linear interpolation for increasing the accuracy in this way it's not clear from the research if calibration cover's the same brand of sensor/chip or if this process needs to be done individually for each sensor.\\ 

Angle Of Arrival/Angle Of Departure (AoA/AoD) allows for the use of angulation are further way's of increasing the accuracy of the location calculation, \citetemp say's this is a particuarly problematic and involves large antenna arrays, which aligns with resources from \citetemp. However combining this method with the lateration method highlighted initially would result in a superior accuracy compared to either individually. Some of the technologies discussed below have the ability to use this method of calculation and in all cases this is discussed the accuracy is better than similar technologies that don't support this IE. Bluetooth 4 vs Bluetooth 5 \citetemp

\subsubsection{Other ways of calculating location}
On top of the above methods of calculation there are also other methods found that can be used for indoor positioning the first of these is by using a devices accelerometer, an interesting paper from \citetemp cover's this system in detail where by accuracy was concluded to be acceptable. The system works by initialising to a known location and then displacing that location based on the read values from the accelerometer. One large consideration for this technology as found within \citetemp research is the more time that passes from the initialisation of the known co-ordinate the more inaccurate the system is, to get around this the user could be promoted periodically to update their location on the map to maintain accuracy. Although no paper's exist about combining this technology with other's it could be viable to use this technology in conjunction with the method above, doing so could allow for estimation of "valid" distance measurements from the sensor's giving the ability to fall back on this and maintain a level of accuracy, doing so could also provide a feedback where the initial co-ordinate is updated via the other system.

Finally there also exists other systems that can be used for indoor positioning however they have large pitfalls not making them viable for this project such as cost, accuracy, dedicated server software etc. some of these technologies include Infrared \citetemp, Ultrasonic \citetemp, RFID \citetemp and Y \citetemp. 

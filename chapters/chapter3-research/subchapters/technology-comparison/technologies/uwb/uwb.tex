\subsubsection{Ultra wide band (UWB)}
UWB is a technology that has existed for a long time and is commonly used for precision positioning scenarios. Although it’s a technology that has been around for a long time it’s something that has only just started becoming available within smart phone tech, from the iPhone 11 and Galaxy note 20 ultra in august 2020. In addition to this google has now also started developing a general API as part of the Android SDK, this will make development easier and more obtainable for developers. This also shows it's potential within the future of smart phone tech and demonstrates proof that over time more smart phones will feature UWB chips.

UWB is responsible for a lot of industrial applications in location tracking such as Sewio (case study below) as well as…… … UWB having implementations in many industry-based applications demonstrates that the technology is promising and works. Sewio’s clients include Volkswagen, Budweiser amongst other’s, this show support of the technology from larger companies. 
UWB like Bluetooth 5.1 has the ability for AOA thus combines the advantages of Bluetooth 5.1. In contrast to Bluetooth 5.1 though UWB using a different spectrum to Bluetooth leading to less interference and a stronger signal. Unfortunately, unlike Bluetooth based implementations UWB is still an expensive technology to implement, Sewio has prices listed on there website and packages at €2850. Part of this costing

A majority of smartphones not having UWB capability’s is a large draw back for the technology and makes the end goal of this project impossible as it will require dedicated hardware for a majority of user’s. Although  this is currently the case it’s important to design the system in mind of expanding technologies, as such the design of the system should allow easy switching of technologies in the future to reflect improvement of smartphone technologies.

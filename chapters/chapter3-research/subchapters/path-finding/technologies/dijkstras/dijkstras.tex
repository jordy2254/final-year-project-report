\subsubsection{Dijkstra's/Uniform cost search}
Dijkstra’s algorithm is a breadth first search algorithm that can traverse weighted graphs and find the shortest path through them to an end location. Unlike A* and Uniform cost search Dijkstra’s spans out from the start point to the end point computing all paths in the process \citetemp. One of the key downsides of this is that Dijkstra’s has to traverse all of the graphs blindly until there is no more solutions available. This computation contains a lot of wasted computing effort and there are better algorithms available \citetemp. Finally traditionally Dijkstras doesn't contain an exit condition and has to have all nodes that are available loaded into memory. This renders Dijkstras only suitable to finite graphs. Due to this trait Dijkstra's also using a large quantity of memory so is not suitable to limited applications or applications where a path to all location's is not needed from the start point.\\
Leading on from Dijkstras is Uniform cost search (UCS) this algorithm focusses on finding the path to the target as opposed to the path to all paths, and allows for the graph to be loaded lazily as the graph is traversed this in turn leads to a lower memory footprint, additionally also makes UCS slightly more efficient, however it's still a slower algorithm to others.
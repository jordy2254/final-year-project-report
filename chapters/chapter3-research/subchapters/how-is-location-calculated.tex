\subsection{How is location calculated}
The following sections aim is to cover the different way location is calculated independent of the technologies required to calculate the values. This will give an insight into the algorithms and code required to turn read values into a final location for the user. 

\subsubsection{Location classification methods}
There currently exists two key types of location system, the first is a classification system this kind of system can classify a zone/area the user is in but not a co-ordinate of their location within that space. An example of this kind of system could be....
The second kind of system is a co-ordinate based system. This method of positioning gives an accurate co-ordinate of the user within real world space. An example implementation of this is GPS which gives the user a latitude and longitude of their location on earth.

With the aim of this project to produce a positioning system for directions the end result will be a co-ordinate based system. For this reason it's important to consider technologies that enable a high range of accuracy. Although if time permits a duel based system could be implemented where small rooms will feature a classification such as that of small offices. Implementing a system like this could save on overall costs of the system making it more accessible.

\subsubsection{General Approaches to Location Calculation}
Calculating a co-ordinate based location is done the same for every technology listed below. This method is called trilateration. Trilateration not to be confused with triangulation, is method of gathering distances from “points” and then finding the intersecting area of those overlapping distances to find the user’s “real location”. The diagram below demonstrates this.
\begin{center}
	\includegraphics[width=10cm]{example-image-a}\\
	Image demonstrating circle intesection for 2d location finding from \citetemp
\end{center}

\subsubsection{Increasing accuracy of distance measurements}
Several of the available technologies listed below rely on the same basic principal of measuring a signal strength. Unfortunately signal strength can suffer issues with accuracy due to a range of issues from localised interference on the sensor. This could background noise, something blocking the sensor and more. In addition different devices on the same standard can broadcast at a different strength which can cause issues with calculating the distance from the sensor as each sensor may need to be calibrated prior to being installed within the system. With a range of different qualities of signal based on the user’s environment and device, it’s important to be able to calibrate sensors to make them more accurate for the user. There exists many papers on this topic and all of them involve calibrating at known distances and the interpolating signal strength from the sensor to the device. 
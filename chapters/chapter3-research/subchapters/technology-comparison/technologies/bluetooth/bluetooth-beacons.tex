\subsubsection{Bluetooth Beacons}
Bluetooth beacons contain 2 main standards these are Eddy-stone \& IBeacon, these standards are developed by google and apple respectively \citetemp. Although they are classified as 2 different standards, they both run on BLE devices and according to several resources such as \citetemp they don’t feature many distinct differences so, for this reason the rest of this section will refer to just beacons as group and not a specific standard of beacons.

Beacons offer one huge advantage that some other technologies don't possess; this huge advantage is the fact they don't rely on networks to set them up, meaning network topologies and other considerations are not needed as in other methods such as Wifi. Beacons having the ability to act as an individual unit unaware of each other also means that it's simple to install and no setup is needed apart from positioning it correctly, this means it can be installed easily by a user of the system and not by an expert cutting initial costs for the install of the system, it also means the system can be expanded and moved as needed. On top of this beacons also use little power running of batteries, some companies such as \citetemp offer beacons that last up to a whopping 58months,  this not only means minimal maintenance of the system is needed but also the system can run in remote areas where electricity and Wi-Fi are not available. Beacons are also readily available and cheap, with available as low as £10 \citetemp this therefore means that initial investment of a system can be done cheaply.

Finally beacons have the potential to offer a high level of privacy because they offer a one way communication to the user's device, this means if the implementation is designed to run locally on the device there is no need for the user to be connected to the internet however if decided the data can be sent back to the server if required. this also means that back-end server systems are not required for the system to work. Although beacons sound very positive there is some limitations of beacons, the key issue is they are built using Bluetooth 4.1 \citetemp this unfortunately means it doesn't have the advantages that are within the next chapter, Bluetooth also run on the open 2.5ghz spectrum the same spectrum that WIFI and other elements run on making them prone to interference meaning a high level of accuracy cannot be achieved. No official resources exist for the obtainable accuracy although \citetemp states it's possible to get an ~50cm accuracy with the system.
